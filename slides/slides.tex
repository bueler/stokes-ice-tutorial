% Copyright 2021  Ed Bueler

\documentclass[10pt,hyperref,dvipsnames]{beamer}

\mode<presentation>{
  \usetheme{Madrid}
  \usecolortheme{beaver}
  \setbeamercovered{transparent}
  \setbeamerfont{frametitle}{size=\large}
}

\setbeamercolor*{block title}{bg=red!10}
\setbeamercolor*{block body}{bg=red!5}

\usepackage[english]{babel}
\usepackage[latin1]{inputenc}
\usepackage{times}
\usepackage[T1]{fontenc}
% Or whatever. Note that the encoding and the font should match. If T1
% does not look nice, try deleting the line with the fontenc.

\usepackage{empheq}
\usepackage{xspace}
\usepackage{verbatim,fancyvrb}

\usepackage{tikz}
\usetikzlibrary{shapes,arrows.meta,decorations.markings,decorations.pathreplacing,fadings,positioning}

\usepackage{hyperref}

% If you wish to uncover everything in a step-wise fashion, uncomment
% the following command:
%\beamerdefaultoverlayspecification{<+->}

\newcommand{\ba}{\mathbf{a}}
\newcommand{\bb}{\mathbf{b}}
\newcommand{\bc}{\mathbf{c}}
\newcommand{\bbf}{\mathbf{f}}
\newcommand{\bg}{\mathbf{g}}
\newcommand{\bn}{\mathbf{n}}
\newcommand{\bq}{\mathbf{q}}
\newcommand{\br}{\mathbf{r}}
\newcommand{\bx}{\mathbf{x}}
\newcommand{\by}{\mathbf{y}}
\newcommand{\bv}{\mathbf{v}}
\newcommand{\bu}{\mathbf{u}}
\newcommand{\bw}{\mathbf{w}}

\newcommand{\bF}{\mathbf{F}}
\newcommand{\bG}{\mathbf{G}}
\newcommand{\bQ}{\mathbf{Q}}

\newcommand{\bzero}{\mathbf{0}}

\newcommand{\grad}{\nabla}
\newcommand{\Div}{\nabla\cdot}
\newcommand{\minmod}{\operatorname{minmod}}

\newcommand{\CC}{\mathbb{C}}
\newcommand{\RR}{\mathbb{R}}

\newcommand{\ddt}[1]{\ensuremath{\frac{\partial #1}{\partial t}}}
\newcommand{\ddx}[1]{\ensuremath{\frac{\partial #1}{\partial x}}}
\newcommand{\Matlab}{\textsc{Matlab}\xspace}
\newcommand{\Octave}{\textsc{Octave}\xspace}
\newcommand{\eps}{\epsilon}

\newcommand{\ip}[2]{\left<#1,#2\right>}

\newcommand{\trefcolumn}[1]{\begin{bmatrix} \phantom{x} \\ #1 \\ \phantom{x} \end{bmatrix}}
\newcommand{\trefmatrixtwo}[2]{\left[\begin{array}{c|c|c} & & \\ #1 & \dots & #2 \\ & & \end{array}\right]}
\newcommand{\trefmatrixthree}[3]{\left[\begin{array}{c|c|c|c} & & & \\ #1 & #2 & \dots & #3 \\ & & & \end{array}\right]}
\newcommand{\trefmatrixgroups}[4]{\left[\begin{array}{c|c|c|c|c|c} & & & & & \\ #1 & \dots & #2 & #3 & \dots & #4 \\ & & & & & \end{array}\right]}

\newcommand{\blocktwo}[4]{\left[\begin{array}{c|c} #1 & #2 \\ \hline #3 & #4 \end{array}\right]}

\newcommand{\ds}{\displaystyle}


\title{Stokes problems using Firedrake}

\subtitle{A tutorial for glaciologists}

\author{Ed Bueler}

\institute[UAF]{University of Alaska Fairbanks}

\date{April 2021}


\begin{document}
\beamertemplatenavigationsymbolsempty

\begin{frame}
  \maketitle
\end{frame}

\begin{frame}
  \frametitle{Outline}

\centerline{\large \href{https://github.com/bueler/stokes-ice-tutorial}{\alert{\texttt{github.com/bueler/stokes-ice-tutorial}}}}

\bigskip
  \tableofcontents[hideallsubsections]
\end{frame}


\section{the problem}

\begin{frame}{the glacier dynamics problem}

\begin{itemize}
\item FIXME show results on velocity and pressure
\end{itemize}
\end{frame}

\section{the equations}

\newcommand{\rhoi}{\rho_{\text{i}}}
\begin{frame}{Glen-Stokes equations}

\begin{columns}
\begin{column}{0.8\textwidth}
\begin{align*}
- \nabla \cdot \tau + \nabla p &= \rhoi \bg &&\text{\emph{stress balance}} \\
\nabla \cdot \bu &= 0 &&\text{\emph{incompressibility}} \\
\tau &= B_n |D\bu|^{(1/n) - 1} D\bu  &&\text{\emph{flow law}}
\end{align*}
\end{column}

\begin{column}{0.2\textwidth}

\vspace{5mm}
%https://www.bl.uk/voices-of-science/interviewees/john-glen#
\hfill \includegraphics[width=0.8\textwidth]{figs/people/jglen.png}
\end{column}
\end{columns}

\bigskip
\begin{itemize}
\item constants: {\small $\rhoi=910 \,\text{kg}\,\text{m}^{-3}$, $g=9.81\,\text{m}\,\text{s}^{-2}$, $n=3$, $B_n=6.8\times 10^7\,\text{Pa}\,\text{s}^{1/3}$}
\item viscosity regularization with $\eps = 10^{-4}$ and $D_0 = 1\,\text{a}^{-1}$:
\begin{equation*}
\nu_\eps(|D\bu|) = \frac{1}{2} B_n \left(|D\bu|^2 + \eps\, D_0^2\right)^{((1/n) - 1)/2}
\end{equation*}
\item combine as system for $\bu,p$:
\begin{align}
- \nabla \cdot \left(2 \nu_\eps(|D\bu|)\, D\bu\right) + \nabla p &= \rhoi \mathbf{g} \label{stokes} \\
\Div \bu &= 0 \label{incompressible}
\end{align}
\end{itemize}
\end{frame}


\begin{frame}{the linear Stokes equations}

\begin{columns}

\begin{column}{0.6\textwidth}
\begin{itemize}
\item if we make viscosity constant ($\nu_0$) then we get the linear Stokes system:
\begin{align*}
- \nabla \cdot \left(2 \nu_0\, D\bu\right) + \nabla p &= \rhoi \mathbf{g}  \\
\Div \bu &= 0
\end{align*}
\item with reformatting:
\begin{align*}
- \nu_0 \nabla^2 \bu + \nabla p &= \rhoi \mathbf{g}  \\
-\Div \bu \qquad \,\,\,\, &= 0
\end{align*}
\item it has block structure
  $$\begin{bmatrix} A & B^\top \\ B & 0 \end{bmatrix}$$
\end{itemize}
\end{column}

\begin{column}{0.4\textwidth}
\hfill \includegraphics[width=0.4\textwidth]{figs/people/gstokes.jpg}

\vspace{10mm}
\hfill \includegraphics[width=0.8\textwidth]{figs/Kstokes.pdf}
\end{column}

\end{columns}
\end{frame}


\section{\texttt{stage1/} \qquad linear Stokes}

\begin{frame}{matrix for Stokes}

\begin{itemize}
\item simplest possible solver for linear stokes is direct method
\item FIXME
\begin{center}
\includegraphics[width=0.2\textwidth]{figs/Kstokes.pdf}
\end{center}
\end{itemize}
\end{frame}


\section{\texttt{stage2/} \qquad Glen-Stokes}

\begin{frame}{Newton's method}

\hfill \includegraphics[width=0.25\textwidth]{figs/people/inewton.jpg}

\vspace{-20mm}
\begin{itemize}
\item FIXME
\end{itemize}
\end{frame}

\begin{frame}{boundary conditions}

\begin{itemize}
\item stress-free top:
\begin{equation*}
\sigma \bn = \left(2 \nu_\eps(|D\bu|) D\bu - pI\right) \bn = \bzero
\end{equation*}
\item no slip base:
\begin{equation*}
\bu = \bzero
\end{equation*}
\end{itemize}
\end{frame}


\begin{frame}{FIXME}

\begin{itemize}
\item FIXME
\end{itemize}
\end{frame}


\section{\texttt{stage3/} \qquad extruded meshes}


\section{\texttt{stage4/} \qquad faster and finer}

\begin{frame}{iterative linear methods}

\hfill \includegraphics[width=0.25\textwidth]{figs/people/akrylov.jpg}

\vspace{-20mm}
\begin{itemize}
\item FIXME
\end{itemize}
\end{frame}

\begin{frame}{multigrid}

\begin{itemize}
\item FIXME
\end{itemize}
\end{frame}


\section{\texttt{stage5/} \qquad 3D}

\begin{frame}{learn more}

\centerline{\large \href{https://github.com/bueler/stokes-ice-tutorial}{\alert{\texttt{github.com/bueler/stokes-ice-tutorial}}}}

\medskip
\hrulefill


\begin{columns}
\begin{column}{0.6\textwidth}
\small

\begin{itemize}
\item Firedrake: \href{https://www.firedrakeproject.org/}{\texttt{firedrakeproject.org}}

    \begin{itemize}
    \scriptsize
    \item tutorials \& manual: \href{https://www.firedrakeproject.org/documentation.html}{\texttt{.../documentation.html}}
    \item Jupyter notebooks page: \href{https://www.firedrakeproject.org/notebooks.html}{\texttt{.../notebooks.html}}
    \end{itemize}
\item PETSc website: \href{https://www.mcs.anl.gov/petsc/}{\texttt{www.mcs.anl.gov/petsc}}
\item for glacier Stokes eqns, see Ch.~1 by Hewitt:


{\scriptsize Fowler \& Ng, ed., \emph{Glaciers and Ice Sheets in the Climate System: The Karthaus Summer School Lecture Notes}, Springer 2021}

\item for finite elements and linear Stokes:

{\scriptsize Elman, Silvester, \& Wathen, \emph{Finite Elements and Fast Iterative Solvers, With Applications in Incompressible Fluid Dynamics}, Oxford 2014, 2nd ed.}

\item for PETSc, Firedrake, and Stokes (Ch.~14):

{\scriptsize Bueler, \emph{PETSc for Partial Differential Equations: Numerical Solutions in C and Python}, SIAM 2021}
\end{itemize}
\end{column}

\begin{column}{0.4\textwidth}

\bigskip
\href{https://www.firedrakeproject.org/}{\includegraphics[width=\textwidth]{figs/firedrakebanner.png}}

\bigskip
\hfill \href{https://www.mcs.anl.gov/petsc/}{\includegraphics[width=0.6\textwidth]{figs/petscbanner.png}}

\vspace{25mm}
\includegraphics[width=0.31\textwidth]{figs/covers/karthaus.png} \hfill
\includegraphics[width=0.31\textwidth]{figs/covers/elman.jpg} \hfill
\href{https://my.siam.org/Store/Product/viewproduct/?ProductId=32850137}{\includegraphics[width=0.32\textwidth]{figs/covers/bueler.jpg}}
\end{column}

\end{columns}
\end{frame}


\end{document}
