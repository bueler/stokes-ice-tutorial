% Copyright 2021-2024  Ed Bueler

\documentclass[10pt,
               hyperref={colorlinks,citecolor=DeepPink4,linkcolor=black,urlcolor=blue},
               svgnames]{beamer}

\mode<presentation>{
  \usetheme{Madrid}
  \usecolortheme{beaver}
  \setbeamercovered{transparent}
  \setbeamerfont{frametitle}{size=\large}
}

\setbeamercolor*{block title}{bg=red!10}
\setbeamercolor*{block body}{bg=red!5}

\usepackage[english]{babel}
\usepackage[latin1]{inputenc}
\usepackage{times}
\usepackage[T1]{fontenc}
% Or whatever. Note that the encoding and the font should match. If T1
% does not look nice, try deleting the line with the fontenc.

\usepackage{amssymb,pifont}
\usepackage{empheq}
\usepackage{xspace}
\usepackage{verbatim,fancyvrb}

\usepackage{tikz}
\usetikzlibrary{shapes,arrows.meta,decorations.markings,decorations.pathreplacing,fadings,positioning}

\usepackage{hyperref}

% If you wish to uncover everything in a step-wise fashion, uncomment
% the following command:
%\beamerdefaultoverlayspecification{<+->}

\newcommand{\ba}{\mathbf{a}}
\newcommand{\bb}{\mathbf{b}}
\newcommand{\bc}{\mathbf{c}}
\newcommand{\bbf}{\mathbf{f}}
\newcommand{\bg}{\mathbf{g}}
\newcommand{\bn}{\mathbf{n}}
\newcommand{\bq}{\mathbf{q}}
\newcommand{\br}{\mathbf{r}}
\newcommand{\bx}{\mathbf{x}}
\newcommand{\by}{\mathbf{y}}
\newcommand{\bv}{\mathbf{v}}
\newcommand{\bu}{\mathbf{u}}
\newcommand{\bw}{\mathbf{w}}

\newcommand{\bF}{\mathbf{F}}
\newcommand{\bG}{\mathbf{G}}
\newcommand{\bQ}{\mathbf{Q}}

\newcommand{\bzero}{\mathbf{0}}

\newcommand{\grad}{\nabla}
\newcommand{\Div}{\nabla\cdot}
\newcommand{\minmod}{\operatorname{minmod}}

\newcommand{\CC}{\mathbb{C}}
\newcommand{\RR}{\mathbb{R}}

\newcommand{\ddt}[1]{\ensuremath{\frac{\partial #1}{\partial t}}}
\newcommand{\ddx}[1]{\ensuremath{\frac{\partial #1}{\partial x}}}
\newcommand{\Matlab}{\textsc{Matlab}\xspace}
\newcommand{\Octave}{\textsc{Octave}\xspace}
\newcommand{\eps}{\epsilon}

\newcommand{\ip}[2]{\left<#1,#2\right>}

\newcommand{\trefcolumn}[1]{\begin{bmatrix} \phantom{x} \\ #1 \\ \phantom{x} \end{bmatrix}}
\newcommand{\trefmatrixtwo}[2]{\left[\begin{array}{c|c|c} & & \\ #1 & \dots & #2 \\ & & \end{array}\right]}
\newcommand{\trefmatrixthree}[3]{\left[\begin{array}{c|c|c|c} & & & \\ #1 & #2 & \dots & #3 \\ & & & \end{array}\right]}
\newcommand{\trefmatrixgroups}[4]{\left[\begin{array}{c|c|c|c|c|c} & & & & & \\ #1 & \dots & #2 & #3 & \dots & #4 \\ & & & & & \end{array}\right]}

\newcommand{\blocktwo}[4]{\left[\begin{array}{c|c} #1 & #2 \\ \hline #3 & #4 \end{array}\right]}

\newcommand{\cmark}{\ding{51}}
\newcommand{\xmark}{\ding{55}}

\newcommand{\rhoi}{\rho_{\text{i}}}

\newcommand{\ds}{\displaystyle}


\title{Stokes problems using Firedrake}

\subtitle{A glacier and ice sheet tutorial}

\author{Ed Bueler}

\institute[UAF]{University of Alaska Fairbanks}

\date[Feb 2024]{version February 2024  \\ \phantom{foo} \\ \phantom{foo} \\
\large \href{https://github.com/bueler/stokes-ice-tutorial}{\texttt{github.com/bueler/stokes-ice-tutorial}} }


\begin{document}
\beamertemplatenavigationsymbolsempty

\begin{frame}
  \maketitle
\end{frame}


\begin{frame}{exclamation points}

\begin{itemize}
\item \alert{I cannot explain everything in an hour!}

\item \alert{ask questions!}

\item \alert{please try the codes!}

\item \alert{send future questions!}
  \begin{itemize}
  \item[$\circ$] by email to \href{mailto:elbueler@alaska.edu}{\texttt{elbueler@alaska.edu}}, or
  \item[$\circ$] using \href{https://github.com/bueler/stokes-ice-tutorial/issues}{github issues}
  \end{itemize}
\end{itemize}
\end{frame}


\setbeamertemplate{section in toc}[default] 

\begin{frame}
  \frametitle{Outline}

\centerline{\large \href{https://github.com/bueler/stokes-ice-tutorial}{\texttt{github.com/bueler/stokes-ice-tutorial}}}

\bigskip
  \tableofcontents[hideallsubsections]
\end{frame}


\section{stage 0 \hspace{9.3mm} the problem and the equations}

\begin{frame}{the glacier \underline{dynamics} problem, without evolution of geometry}

\begin{itemize}
\item conservation principles apply to ice sheets:
    \begin{itemize}
    \item[$\circ$] mass \hfill \cmark\qquad today
    \item[$\circ$] momentum \hfill \cmark\qquad today
    \item[$\circ$] energy \hfill \xmark\,\, \emph{not today}
    \end{itemize}
\item Stokes model $=$ momentum conservation $+$ incompressibility
    \begin{itemize}
    \item[$\circ$] incompressibility is one aspect of mass conservation
    \item[$\circ$] no attempt \emph{today} to model geometry changes using surface balance, the other aspect of mass conservation
    \end{itemize}
\item \alert{the Stokes model determines velocity and pressure from given geometry}

\bigskip
{\scriptsize velocity} \hfill \includegraphics[width=0.85\textwidth]{figs/speed.png}

\bigskip
{\scriptsize pressure} \hfill \includegraphics[width=0.85\textwidth]{figs/pressure.png}

\bigskip
\end{itemize}
\end{frame}


\begin{frame}{Glen-Stokes equations}

\begin{columns}
\begin{column}{0.8\textwidth}
\begin{align*}
- \nabla \cdot \tau + \nabla p &= \rhoi \bg &&\text{\emph{stress balance}} \\
\nabla \cdot \bu &= 0 &&\text{\emph{incompressibility}} \\
\tau &= B_n |D\bu|^{(1/n) - 1} D\bu  &&\text{\emph{Glen flow law}}
\end{align*}
\end{column}

\begin{column}{0.2\textwidth}

\vspace{5mm}
%https://www.bl.uk/voices-of-science/interviewees/john-glen#
\hfill \includegraphics[width=0.6\textwidth]{figs/people/jglen.png}

\hfill {\tiny John Glen}
\end{column}
\end{columns}

\medskip
\begin{itemize}
\item $\bu$ is velocity, $p$ is pressure, $\tau$ is the deviatoric stress tensor, and
  $$D\bu=\tfrac{1}{2}\left(\grad \bu + \grad \bu^\top\right)$$
is the strain-rate tensor
\item constants: {\small $\rhoi=910 \,\text{kg}\,\text{m}^{-3}$, $g=9.81\,\text{m}\,\text{s}^{-2}$, $n=3$, $B_n=6.8\times 10^7\,\text{Pa}\,\text{s}^{1/3}$}
    \begin{itemize}
    \item[$\circ$] isothermal, so $B_n$ is constant
    \end{itemize}
\item now regularize viscosity using $\eps = 10^{-4}$ and $D_0 = 1\,\text{a}^{-1}$:
\begin{equation*}
\nu_\eps(|D\bu|) = \tfrac{1}{2} B_n \left(|D\bu|^2 + \eps\, D_0^2\right)^{((1/n) - 1)/2}
\end{equation*}
\item eliminate $\tau$ to give system for $\bu,p$:
\begin{align*}
- \nabla \cdot \left(2 \nu_\eps(|D\bu|)\, D\bu\right) + \nabla p &= \rhoi \mathbf{g} \\
\Div \bu &= 0
\end{align*}
\end{itemize}
\end{frame}


\begin{frame}{boundary conditions, as simple as possible}

\begin{itemize}
\item stress-free top:
\begin{equation*}
\sigma \bn = \left(2 \nu_\eps(|D\bu|) D\bu - pI\right) \bn = \bzero
\end{equation*}

\bigskip
\includegraphics[width=0.9\textwidth]{figs/speed.png}

\bigskip
\item no slip base:
\begin{equation*}
\bu = \bzero
\end{equation*}
    \begin{itemize}
    \item[$\circ$] no attempt \emph{today} to model sliding
    \end{itemize}
\end{itemize}
\end{frame}


\begin{frame}{the linear Stokes equations}

\begin{columns}

\begin{column}{0.6\textwidth}
\begin{itemize}
\item if we make viscosity constant ($\nu_0$) then we get a linear Stokes system
\begin{align*}
- \nabla \cdot \left(2 \nu_0\, D\bu\right) + \nabla p &= \rhoi \mathbf{g}  \\
\Div \bu &= 0
\end{align*}
\item with reformatting, using \quad $2 \nabla \cdot D\bu = \nabla^2 \bu$
\begin{align*}
- \nu_0 \nabla^2 \bu + \nabla p &= \rhoi \mathbf{g}  \\
-\Div \bu \qquad \,\,\,\, &= 0
\end{align*}
\item which has symmetric block structure
  $$\begin{bmatrix} A & B^\top \\ B & 0 \end{bmatrix} \begin{bmatrix} \bu \\ p  \end{bmatrix} = \begin{bmatrix} \bbf \\ 0 \end{bmatrix}$$
\end{itemize}
\end{column}

\begin{column}{0.4\textwidth}
\hfill \includegraphics[width=0.4\textwidth]{figs/people/gstokes.jpg}

\hfill {\tiny George Stokes}

\vspace{7mm}
\hfill \includegraphics[width=0.8\textwidth]{figs/Kstokes.pdf}
\end{column}

\end{columns}
\end{frame}


\begin{frame}{finite elements \dots with no details}

\begin{itemize}
\item exact solution of the Stokes model is impossible for most geometries
    \begin{itemize}
    \item[$\circ$] \emph{exercise.}  slab-on-a-slope is an exception
    \end{itemize}
\item so we solve numerically using the finite element (FE) method
\item in summary, the FE method allow us to start from a mesh of triangles {\small (or quadrilaterals, prisms, \dots) over our domain, {\footnotesize then express the approximate solution in terms of functions built from the mesh, {\scriptsize then more and more details which I am skipping \dots}}}

\medskip
\includegraphics[width=0.2\textwidth]{figs/hat.png} \hfill

\vspace{-9mm}
\hfill \includegraphics[width=0.7\textwidth]{figs/stage2.png}

\bigskip
\item \alert{the whole point of Firedrake is to \emph{not}\, see the details of FE}
\item one needs to know: the first step is to write a weak form
\end{itemize}
\end{frame}


\begin{frame}[fragile]
\frametitle{weak form}

\begin{itemize}
\item start from the original equations, also called the \emph{strong form}:
\small
\begin{align}
- \nabla \cdot \left(2 \nu_\eps(|D\bu|)\, D\bu\right) + \nabla p &= \rhoi \mathbf{g} \\
\Div \bu &= 0
\end{align}
\normalsize
\item multiply (1) by test velocity $\bv$ and (2) by test pressure $q$, integrate by parts, and combine into one nonlinear residual function:
    $$F(\bu,p;\bv,q) = \int_{\Omega} 2 \nu_\eps(|D\bu|)\, D\bu\,:\,D\bv - p \nabla \cdot \bv - q \nabla \cdot \bu - \rhoi \mathbf{g} \cdot \bv \,dx$$
\item in Firedrake's language (= \emph{Unified Form Language}) it looks like:

\medskip
\begin{Verbatim}[fontsize=\small]
fbody = rho * Constant((0.0, 0.0, - g))
Du2 = 0.5 * inner(D(u), D(u)) + (eps * Dtyp)**2.0
nu = 0.5 * Bn * Du2**((1.0 / n - 1.0)/2.0)
F = ( inner(2.0 * nu * D(u), D(v)) \
      - p * div(v) - q * div(u) - inner(fbody, v) ) * dx
\end{Verbatim}

\medskip
\item the equation solved by the FE method is the statement
   $$F(\bu,p;\bv,q) = 0 \qquad \text{ for all } \bv \text{ and } q$$
\end{itemize}
\end{frame}


\begin{frame}[fragile]
\frametitle{mixed finite elements for fluid problems}

\begin{itemize}
\item the other details we need to know about:
    \begin{itemize}
    \item[{\color{black} 1.}] \alert{choose separate function spaces for the velocity and the pressure}
    \item[{\color{black} 2.}] \alert{choose spaces that the experts say are ``stable''}
    \end{itemize}

\bigskip
\item Firedrake makes choosing $P_2 \times P_1$ (Taylor-Hood) elements very easy:

\medskip
\begin{Verbatim}[fontsize=\small]
V = VectorFunctionSpace(mesh, 'Lagrange', 2)   # u space
W = FunctionSpace(mesh, 'Lagrange', 1)         # p space
Z = V * W
up = Function(Z)
u, p = split(up)
v, q = TestFunctions(Z)
\end{Verbatim}

\medskip
    \begin{itemize}
    \item[$\circ$] other mixed spaces are just as easy
    \end{itemize}
\end{itemize}
\end{frame}


\AtBeginSection[]
{
  \begin{frame}{Outline}
  \centerline{\large \href{https://github.com/bueler/stokes-ice-tutorial}{\texttt{github.com/bueler/stokes-ice-tutorial}}}

  \bigskip
  \tableofcontents[currentsection]
  \end{frame}
}

\section{\texttt{stage1/} \qquad linear Stokes}

\begin{frame}[fragile]
\frametitle{running Firedrake programs}

\begin{itemize}
\item now I'm actually going to run some code
\item grab my tutorial (codes and these slides):

\medskip
\begin{Verbatim}[fontsize=\small]
$ git clone https://github.com/bueler/stokes-ice-tutorial.git
\end{Verbatim}
%$

\medskip
\item install Firedrake following these directions:

\begin{center}
\href{https://www.firedrakeproject.org/download.html}{\texttt{firedrakeproject.org/download.html}}
\end{center}
\item activate the Python virtual environment every time you use Firedrake:

\medskip
\begin{Verbatim}[fontsize=\small]
$ unset PETSC_DIR; unset PETSC_ARCH;  # may be needed
$ source ~/firedrake/bin/activate
\end{Verbatim}

\medskip
    \begin{itemize}
    \item[$\circ$] I use an alias \texttt{drakeme} for this
    \end{itemize}
\item other tools I will need:
    \begin{itemize}
    \item[$\circ$] Gmsh \quad \href{https://gmsh.info/}{\texttt{gmsh.info}}
    \item[$\circ$] Paraview \quad \href{https://www.paraview.org/}{\texttt{paraview.org}}
    \end{itemize}
\end{itemize}
\end{frame}


\begin{frame}[fragile]
\frametitle{\texttt{stage1/} \qquad linear Stokes}

\begin{itemize}
\item \emph{purpose:} solve linear Stokes on a trapezoidal glacier

    \begin{itemize}
    \item[$\circ$] simplified weak form
    \end{itemize}
{\small
    $$F(\bu,p;\bv,q) = \int_{\Omega} 2 \nu_0\, D\bu\,:\,D\bv - p \nabla \cdot \bv - q \nabla \cdot \bu - \rhoi \mathbf{g} \cdot \bv \,dx$$
}
\item \emph{source files:} \texttt{domain.geo}, \texttt{solve.py} \hfill \alert{$\gets$ inspect these!}
\item \emph{generated files:} \texttt{domain.msh}, \texttt{domain.pvd}
\end{itemize}

\bigskip
\begin{Verbatim}
$ cd stage1/
$ gmsh -2 domain.geo             # mesh the domain
$ gmsh domain.msh                # view the mesh
$ python3 solve.py               # solve linear Stokes
$ paraview domain.pvd            # view the solution
\end{Verbatim}
%$

\vspace{5mm}
\begin{center}
\includegraphics[width=0.9\textwidth]{figs/stage1.png}

{\tiny speed $|\bu|$}
\end{center}
\end{frame}


\section{\texttt{stage2/} \qquad Glen-Stokes}

\begin{frame}{Newton's method}

\begin{columns}
\begin{column}{0.8\textwidth}
\begin{itemize}
\item linear Stokes \small (\texttt{stage1/}) \normalsize needs only a single linear system:
  $$\begin{bmatrix} A & B^\top \\ B & 0 \end{bmatrix} \begin{bmatrix} \bu \\ p  \end{bmatrix} = \begin{bmatrix} \bbf \\ 0 \end{bmatrix}$$
    \begin{itemize}
    \item[$\circ$] Firedrake asks PETSc's KSP component to solve this
    \end{itemize}
\item but for Glen-Stokes the weak form is nonlinear in $\bu$:
\small
$$F(\bu,p;\bv,q) = \int_{\Omega} 2\, \alert{\nu_\eps(|D\bu|)\, D\bu}\,:\,D\bv - p \nabla \cdot \bv - q \nabla \cdot \bu - \rhoi \mathbf{g} \cdot \bv \,dx$$
\normalsize
\item Firedrake automatically applies Newton's method if we write the problem as \texttt{solve(F == 0,\dots)}
\item all you need to know about Newton's iteration:
    \begin{itemize}
    \item[$\circ$] it does linearization around each iterate
    \item[$\circ$] Firedrake uses UFL symbolic differentiation for linearization
    \item[$\circ$] Firedrake asks PETSc's SNES to do the Newton iteration
    \item[$\circ$] SNES options will monitor and control the Newton iteration
    \end{itemize}
\end{itemize}
\end{column}
\begin{column}{0.2\textwidth}
\vspace{25mm}

\hfill \includegraphics[width=0.8\textwidth]{figs/people/inewton.jpg}

\hfill {\tiny Isaac Newton}
\end{column}
\end{columns}
\end{frame}


\begin{frame}[fragile]
\frametitle{\texttt{stage2/} \qquad Glen-Stokes}

\begin{itemize}
\item \emph{purpose:} solve Glen-Stokes on a flat-bed glacier
\item \emph{source files:} \texttt{domain.py}, \texttt{solve.py} \hfill \alert{$\gets$ inspect these!}
\item \emph{generated files:} \texttt{dome.geo}, \texttt{dome.msh}, \texttt{dome.pvd}
\end{itemize}

\bigskip
\begin{Verbatim}
$ cd stage2/
$ python3 domain.py
$ gmsh -2 dome.geo
$ gmsh dome.msh
$ python3 solve.py
$ paraview dome.pvd
\end{Verbatim}

\vspace{6mm}
\begin{center}
\includegraphics[width=0.95\textwidth]{figs/stage2.png}

{\tiny speed $|\bu|$}
\end{center}
\end{frame}


\section{\texttt{stage3/} \qquad extruded meshes}

\begin{frame}[fragile]
\frametitle{\texttt{stage3/} \qquad extruded meshes}

\begin{itemize}
\item \emph{purpose:} solve same problem using an extruded quadrilateral mesh
\item \emph{source files:} \texttt{solve.py} \hfill \alert{$\gets$ inspect this!}
\item \emph{generated files:} \texttt{dome.pvd}
\end{itemize}

\bigskip
\begin{Verbatim}
$ cd stage3/
$ python3 solve.py
$ paraview dome.pvd
\end{Verbatim}
%$

\vspace{10mm}
\begin{center}
\includegraphics[width=0.9\textwidth]{figs/stage3.png}

{\tiny speed $|\bu|$}
\end{center}
\end{frame}


\begin{frame}[fragile]
\frametitle{convergence?}

\begin{itemize}
\item \texttt{stage3/} code \texttt{solve.py} allows adjustable resolution, e.g.:
\begin{Verbatim}
$ ./solve.py -mx 40 -mz 4
\end{Verbatim}
%$
\item velocity results are reasonable and consistent:

\medskip
% timer python3 solve.py -mx MX -mz MZ
\begin{center}
\small
\begin{tabular}{c|rrr}
mesh           &   $\Delta x\times \Delta z$ (m) & av.~$|\bu|$ (m/a) & max.~$|\bu|$ (m/a) \\ \hline
$40\times 4$   &   $500 \times 250$ &        1788 &         3272 \\
$80\times 8$   &   $250 \times 125$ &        1769 &         3215 \\
$160\times 16$ &    $125 \times 63$ &        1762 &         3197 \\
$320\times 32$ &     $63 \times 31$ &        1759 &         3191 \\
$640\times 64$ &     $31 \times 16$ &        1758 &         3189
\end{tabular}
\normalsize
\end{center}

\item quantities of interest are seemingly converging
\item however: we do not know the exact solution and we cannot check for actual numerical convergence
\end{itemize}
\end{frame}


\section{\texttt{stage4/} \qquad bells and whistles}

\begin{frame}[fragile]
\frametitle{\texttt{stage4/} \qquad bells and whistles}

\begin{itemize}
\item \emph{purpose:} additional robust and/or useful features
    \begin{itemize}
    \item[$\circ$] rescale the equations  \hfill \emph{better conditioning}
    \item[$\circ$] vertical grid sequencing  \hfill \emph{better initial iterates}
    \item[$\circ$] $100\eps$ on coarse meshes in sequencing  \hfill \emph{better initial iterates}
    \item[$\circ$] generate stress tensor $\tau$ from solution  \hfill \emph{diagnostic}
    \item[$\circ$] generate effective viscosity $\nu_\eps$ from solution  \hfill \emph{diagnostic}
    \end{itemize}
\item \emph{source files:} \texttt{solve.py} \hfill \alert{$\gets$ inspect this!}
\item \emph{generated files:} \texttt{dome.pvd}
\end{itemize}

\bigskip
\begin{Verbatim}
$ cd stage4/
$ python3 solve.py
$ python3 solve.py -mx 320 -mz 2 -refine 2
$ paraview dome.pvd
\end{Verbatim}
%$

\bigskip
\begin{center}
\includegraphics[width=0.9\textwidth]{figs/stage4.png}

{\tiny deviatoric stress magnitude $\|\tau\|$}
\end{center}
\end{frame}


\section{\texttt{stage5/} \qquad 3D glaciers in parallel}

\begin{frame}[fragile]
\frametitle{\texttt{stage5/} \qquad 3D glaciers in parallel}

\begin{itemize}
\item \emph{purpose:} 3D ice sheet in parallel on bumpy bed
\item \emph{source files:} \texttt{solve.py} \hfill \alert{$\gets$ inspect this!}
\item \emph{generated files:} \texttt{dome.pvd}
\end{itemize}

\bigskip
\begin{Verbatim}
$ cd stage5/
$ mpiexec -n 4 python3 solve.py
$ paraview dome.pvd              # see "rank" field
$ mpiexec -n 4 python3 solve.py -o finer.pvd \
    -refine 2 -baserefine 3
$ paraview finer.pvd
\end{Verbatim}
%$

\bigskip
\begin{center}
\includegraphics[width=0.4\textwidth]{figs/stage5.png}
\end{center}
\end{frame}


\begin{frame}[fragile]
\frametitle{solver performance limitations}

\begin{itemize}
\item the run at bottom is about as far as I can refine on my big desktop
    \begin{itemize}
    \item[$\circ$] 100 GB ram, plenty of cores
    \item[$\circ$] 38 minutes run time
    \end{itemize}
\item \alert{\emph{limiting factor:} the direct solver for each Newton step linear system uses too much memory when generating the LU factors}
    \begin{itemize}
    \item[$\circ$] direct solvers experience \emph{fill-in}, especially in 3D, and they are slow
    \item[$\circ$] {\scriptsize E.~Bueler (2023). \emph{Performance analysis of high-resolution ice-sheet simulations}, J.~Glaciol.~69~(276), 930--935\, \href{https://doi.org/10.1017/jog.2022.113}{10.1017/jog.2022.113}}
    \end{itemize}
\item for higher resolution we need a scalable solver: matrix-free, Schur-complement preconditioning, multigrid on the $\bu$-$\bu$ block
    \begin{itemize}
    \item[$\circ$] {\scriptsize T.~Isaac, G.~Stadler, \& O.~Ghattas (2015). \emph{Solution of nonlinear Stokes equations discretized by high-order finite elements on nonconforming and anisotropic meshes, with application to ice sheet dynamics}, SIAM J.~Sci.~Comput.~37 (6), B804--B833, \href{https://doi.org/10.1137/140974407}{10.1137/140974407}}
    \end{itemize}
\item scalable solvers = another talk!
\end{itemize}

\bigskip
\begin{Verbatim}
$ mpiexec -n 12 python3 solve.py -o finest.pvd \
    -refine 2 -baserefine 4 -s_snes_atol 1.0e-2
\end{Verbatim}
%$
\end{frame}


\begin{frame}{learn more}

\centerline{\large \href{https://github.com/bueler/stokes-ice-tutorial}{\texttt{github.com/bueler/stokes-ice-tutorial}}}

\medskip
\hrulefill


\begin{columns}
\begin{column}{0.6\textwidth}
\small

\begin{itemize}
\item Firedrake: \href{https://www.firedrakeproject.org/}{\texttt{firedrakeproject.org}}

    \begin{itemize}
    \scriptsize
    \item tutorials \& manual: \href{https://www.firedrakeproject.org/documentation.html}{\texttt{.../documentation.html}}
    \item Jupyter notebooks page: \href{https://www.firedrakeproject.org/notebooks.html}{\texttt{.../notebooks.html}}
    \end{itemize}
\item PETSc: \href{https://petsc.org}{\texttt{petsc.org}}
\item for Glen-Stokes eqns, see Ch.~1 by Hewitt:


{\scriptsize Fowler \& Ng, ed., \emph{Glaciers and Ice Sheets in the Climate System: The Karthaus Summer School Lecture Notes}, Springer 2021}

\item for finite elements and linear Stokes:

{\scriptsize Elman, Silvester, \& Wathen, \emph{Finite Elements and Fast Iterative Solvers, With Applications in Incompressible Fluid Dynamics}, Oxford 2014, 2nd ed.}

\item for PETSc, Firedrake, and Stokes (Ch.~14):

{\scriptsize Bueler, \emph{PETSc for Partial Differential Equations: Numerical Solutions in C and Python}, SIAM 2021}
\end{itemize}
\end{column}

\begin{column}{0.4\textwidth}

\bigskip
\href{https://www.firedrakeproject.org/}{\includegraphics[width=\textwidth]{figs/firedrakebanner.png}}

\bigskip
\hfill \href{https://www.mcs.anl.gov/petsc/}{\includegraphics[width=0.6\textwidth]{figs/petscbanner.png}}

\vspace{15mm}
\includegraphics[width=0.31\textwidth]{figs/covers/karthaus.png} \hfill
\includegraphics[width=0.31\textwidth]{figs/covers/elman.jpg} \hfill
\href{https://my.siam.org/Store/Product/viewproduct/?ProductId=32850137}{\includegraphics[width=0.32\textwidth]{figs/covers/bueler.jpg}}
\end{column}

\end{columns}
\end{frame}


\end{document}
